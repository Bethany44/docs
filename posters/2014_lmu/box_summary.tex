\small
Python combines the power of a \textbf{full-blown programming language} with the
flexibility and fast code development of an \textbf{interactive scripting
language}. Its extensive standard library and large variety of freely available
high quality scientific modules cover most needs in \textbf{developing
scientific processing workflows}.

\textbf{ObsPy extends Python’s capabilities to fit the specific needs that
arise when working with seismological data.} It \textit{a)} provides
\textbf{read and write support for all of the most important waveform, station
and event metadata formats} \textit{b)} enables
\textbf{direct access to all important data centers, web services and
databases} to easily retrieve waveform data and station/event metadata and
\textit{c)} comes with a continuously growing \textbf{powerful signal
processing toolbox} that covers all every-day tasks in seismological analysis.

In combination with mature and free Python packages like NumPy, SciPy,
Matplotlib, IPython, Pandas and PyQt, \textbf{ObsPy makes it possible to
develop complete seismological processing workflows}, ranging
from reading locally stored data or requesting data from one or more different
data centers via signal analysis and data processing to visualization in GUI
and web applications, output of modified/derived data and the creation of
publication-quality figures.

All functionality is \textbf{extensively documented} and the online \textbf{ObsPy
Tutorial and Gallery} give a good impression of the wide range of possible use
cases. ObsPy is tested and \textbf{running on Linux, MacOS X and Windows} and
comes with installation routines for these systems. ObsPy is developed in a
test-driven approach and is available under the \textbf{LGPLv3 open source
licence}.

Users are welcome to request help, report bugs, propose enhancements or
contribute code via either the user mailing list or the \textbf{project page on
GitHub}.
