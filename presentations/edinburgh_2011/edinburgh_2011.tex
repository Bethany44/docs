\documentclass[t,10pt,compress=false,usepdftitle=false]{beamer}

\usetheme[compress]{LMU}
\setbeamercovered{dynamic} % shows items in white grey before active

%no navigational bars:
\setbeamertemplate{navigation symbols}{}

\usepackage[utf8x]{inputenc}
\usepackage[ngerman]{babel}
\usepackage{listings}
\usepackage{courier}
\usepackage{xcolor}
\usepackage{verbatim}
\usepackage{latexsym}

% http://stackoverflow.com/questions/1662037/how-to-write-programming-code-containing-the-character-in-latex
\makeatletter
\let \@sverbatim \@verbatim
\def \@verbatim {\@sverbatim \verbatimplus}
{\catcode`'=13 \gdef \verbatimplus{\catcode`'=13 \chardef '=13 }} 
\makeatother

% -----------------------------------------------------------------------------
%
%\newtheorem{definition}{Definition}
\newcommand{\foot}[1]{_{\mbox{\footnotesize #1}}}
\newcommand{\head}[1]{^{\mbox{\footnotesize #1}}}
%
%
\newcommand{\ones}{\mathbb{I}}
\newcommand{\nat}{\mathbb{N}}
\newcommand{\real}{\mathbb{R}}
\newcommand{\ganz}{\mathbb{Z}}
%
%
\newcommand{\RRE}{\mbox{RRE}}
\newcommand{\nnz}[1]{\mbox{nnz}(#1)}
\newlength{\Hoehe}
\renewcommand{\vec}[1]{#1}
\newlength{\GLaenge}
\setlength{\GLaenge}{3.5cm}
%
%
\definecolor{MyGrey}{gray}{0.45}
\def\bstheta{\boldsymbol{\theta}}
\def\bsalpha{\boldsymbol{\alpha}}
\def\bsk{\boldsymbol{k}}
\def\bsx{\boldsymbol{x}}
\def\bsh{\boldsymbol{h}}
%
% Centred minipage environment
%
\newenvironment{cmpage}[1]{
\begin{center}
\begin{minipage}{#1\textwidth}}%
{\end{minipage}\end{center}}
%
%
\newcommand{\POS}{\color{blue}\item [\boldmath{$+$}]}
\newcommand{\NEG}{\color{red}\item [{\boldmath$-$}]}
\newcommand{\NTR}{\color{black}\item [$\circ$]}
\newcommand{\f}[1]{\mathfrak{#1}}
\newcommand{\old}{^{\mbox{\small \color{blue} old}}}
\newcommand{\new}{^{\mbox{\small \color{red} new}}}
\newcommand{\diag}[1]{\mbox{diag}\left(#1\right)}
%
%
\newcommand{\myBlank}{\textvisiblespace}
\newcommand{\noSpace}{\makebox[0pt]{\quad}}
%
% Old style colour commands
%
\newcommand{\CB}{\color{blue}}
\newcommand{\CR}{\color{red}}
\newcommand{\CG}{\color{green}}
\newcommand{\CC}{\color{cyan}}
%
\definecolor{myWhite}{rgb}{1.00,1.00,1.00}  % real white
\definecolor{myGrey}{rgb}{0.78,0.83,0.94}   % 'light grey blue'
\definecolor{myYellow}{rgb}{1.00,1.00,0.00} % yellow
\definecolor{myOrange}{rgb}{1.00,0.65,0.00} % orange
\definecolor{myCyan}{rgb}{0.00,1.00,1.00}   % cyan
%
% Some abbrevs for setting brief code parts
%
\newcommand{\ttA}{\mbox{\texttt{A}}}
\newcommand{\ttB}{\mbox{\texttt{B}}}
\newcommand{\ttC}{\mbox{\texttt{C}}}
\newcommand{\ttD}{\mbox{\texttt{D}}}
\newcommand{\code}[1]{\mbox{\texttt{#1}}}
\newcommand{\ccode}[1]{\cemphd{\texttt{#1}}}
%
% Commands for slides taken from 'Insides'
%
\newcommand{\rst}{\textcolor{emphcolora}{\ast}}
\newcommand{\bst}{\textcolor{emphcolorb}{\ast}}
%
%
%
\definecolor{textcolor} {rgb}{0,0,0}
\definecolor{decocolor} {rgb}{0,0,0}
\definecolor{emphcolora}{rgb}{1,0,0}              % pure red
\definecolor{emphcolorb}{rgb}{0,0,1}              % pure blue
\definecolor{emphcolorc}{cmyk}{0,1,0,0}           % pure magenta
%\definecolor{emphcolord}{cmyk}{0.64,0,0.95,0.20} % sort of green
\definecolor{emphcolord}{rgb}{0,0.4,0.12}         % same as lmu@darkgreen
\definecolor{emphcolore}{cmyk}{1,0,0,0}           % pure cyan
\definecolor{linkcolor} {rgb}{0,0,0}
%
% Commands emphasising text using color
%
\newcommand{\cempha}[1]{{\color{emphcolora}#1}}
\newcommand{\cemphb}[1]{{\color{emphcolorb}#1}}
\newcommand{\cemphc}[1]{{\color{emphcolorc}#1}}
\newcommand{\cemphd}[1]{{\color{emphcolord}#1}}
\newcommand{\cemphe}[1]{{\color{emphcolore}#1}}
\newcommand{\cemphf}[1]{{\color{decocolor}#1}}
% -----------------------------------------------------------------------------
% myColorBox
% -----------------------------------------------------------------------------
\setbeamercolor{myBoxColor}{fg=black,bg=white}
\setbeamercolor{myBoxColorHead}{fg=red,bg=white}
% \newenvironment{myColorBox}[2]{%
% \begin{beamerboxesrounded}[shadow=true,lower=myBoxColor,upper=myBoxColorHead,
% width=#1\textwidth]{#2}}%
% {\end{beamerboxesrounded}}
\newenvironment{myColorBox}[2]{%
\begin{cmpage}{#1}%
\begin{beamerboxesrounded}[shadow=true,lower=myBoxColor,upper=myBoxColorHead]%
{#2}}%
{\end{beamerboxesrounded}\end{cmpage}}
%
% -----------------------------------------------------------------------------
% Math Operators, alternate greek symbols and the like
% -----------------------------------------------------------------------------
\DeclareMathOperator{\grad}{grad}
\DeclareMathOperator{\mydiv}{div}
\DeclareMathOperator{\Grad}{grad}
\DeclareMathOperator{\Div}{div}
%\newcommand{\grad}{\mbox{grad}}
%\newcommand{\mydiv}{\mbox{div}}
\renewcommand{\rho}{\varrho}
%
% -----------------------------------------------------------------------------
% Some color defintions to be compatible with XFIG
% -----------------------------------------------------------------------------
%
\definecolor{XFIGgold}{rgb}{1.00,0.84,0.00}
\definecolor{XFIGltblue}{rgb}{0.53,0.81,1.00}
\definecolor{XFIGred}{rgb}{1.00,0.00,0.00}
% -----------------------------------------------------------------------------


\hypersetup{pdfpagemode=FullScreen}
\hypersetup{breaklinks=false}
\hypersetup{colorlinks=true}
\hypersetup{urlcolor=blue}

\title[]{\parbox[c][][c]{0.7\paperwidth}{\centering Seismological software developments at LMU Munich: Python \& ObsPy}}
\author[]{Robert Barsch, Tobias Megies}
\date[]{2011-03-08}
\institute{Department für Geo- and Umweltwissenschaften (Geophysik)\\ Ludwig-Maximilians-Universit\"at M\"unchen}

\begin{document}
\lstset{%
language=Python,          % choose the language of the code
basicstyle=\footnotesize, % the size of the fonts that are used for the code
numbers=none,             % where to put the line-numbers
numberstyle=\footnotesize,% the size of the fonts that are used for the line-numbers
stepnumber=1,             % step between line-numbers. For 1 each line will be numbered
numbersep=5pt,            % how far the line-numbers are from the code
%backgroundcolor=\color{white}, % choose the background color.
showspaces=false,         % show spaces adding particular underscores
showstringspaces=false,   % underline spaces within strings
showtabs=false,           % show tabs within strings adding particular underscores
frame=none,               % e.g. single, adds a frame around the code
tabsize=2,                % sets default tabsize to 2 spaces
captionpos=b,             % sets the caption-position to bottom
breaklines=true,          % sets automatic line breaking
breakatwhitespace=false,  % sets if automatic breaks should only happen at whitespace
escapeinside={\%*}{*)},    % if you want to add a comment within your code
%keywordstyle=\color{red}\bfseries\emph,
}
\maketitle

%%
\begin{frame}[fragile]
    \frametitle{SeisHub}
    Native, document-centric XML database
    \begin{itemize}
        \item RESTful Web service (HTTP, HTTPS), 
        \item Standard relational database as back-end
        \item Both worlds: SQL for querying and manipulating data and any standard connected to XML, e.g. XSLT or XSD
        \item Not restricted to seismology at all
    \end{itemize}
    Extended to a ''classical'' seismic database
    \begin{itemize}
        \item Index of local, file-based waveform archive (MiniSEED, GSE2, SAC, ...)
        \begin{itemize}
            \item Meta Data: Gaps, overlaps, quality and timing information
            \item Waveform previews (30s)
        \end{itemize}
        \item XML resource types for handling inventory data (XML-SEED) and events (QuakeML based) 
        \item Remote waveforms access (ArcLink)
    \end{itemize}
\end{frame}

%%
\begin{frame}[fragile]
    \frametitle{SeisHub: Technical Details}
    \begin{itemize}
        \item Python-based, standalone web service
        \item Platform independent, open source (GPL)
        \item Implementation of various web protocols, like HTTP, SSH, SFTP
        \item Plug-in architecture: Dynamic discovering and loading of modules and support for Python .egg files
        \item Development remarks:
        \begin{itemize}
            \item Test-driven development proven software, so far $\Rightarrow$ ca. 250 test cases
            \item Well-documented source code
            \item Subversion
            \item Trac: ticket system and project wiki
        \end{itemize}
    \end{itemize}
\end{frame}

%%
\begin{frame}[fragile]
    \frametitle{SeisHub: Database Design}
    Data storage
    \begin{itemize}
        \item Primary data $\Rightarrow$ file system
        \begin{itemize}
            \item Continuous waveform archive (MiniSEED, GSE2, SAC ...)
            \item Other data via (GeoTIFF, GPS time series, etc.) file system
        \end{itemize}
        \item Meta Data $\Rightarrow$ Web service on top of a XML/relational database hybrid
        \begin{itemize}
            \item Data is packed into a XML document $\Rightarrow$ Data structure is within the document, no need for a predefined database schema
            \item XML resources are archived into a BLOB field
            \item Only searchable values are indexed
            \item Pointers to primary data
        \end{itemize}
    \end{itemize}
\end{frame}

%%
\begin{frame}[fragile]
    \frametitle{SeisHub: Database Design}
    Data access
    \begin{itemize}
        \item HTTP/HTTPS: REST web service
        \begin{itemize}
            \item XML documents have a fixed resource identifier (URL's)
            \item Data transformation via XML Style Sheets on request (?output=...)
            \item Data validation via Schema (XML Schema, RelaxNG, Schematron) on resource upload
            \item Document properties like related meta data or indexes
        \end{itemize}
        \item SFTP: XML documents mapped into a virtual file system
    \end{itemize}
\end{frame}

%%
\begin{frame}[fragile]
    \frametitle{SeisHub: Database Design}
    \begin{itemize}
        \item Indexing
        \begin{itemize}
            \item Generated using a XPath expression, type and additional options
            \item Simple creation + reindexing via web interface
            \item Various build-in types (datetime, bool, numeric, double, float, etc..)
            \item ProcessorIndex: custom processing
        \end{itemize}
        \item Searching
        \begin{itemize}
            \item XPath-like query on XML catalog object (restricted to indexes)
            \item SQL on database object
        \end{itemize}
        \item Mapper: predefined queries \& output format bound to an fixed URL
        \item FileSystemResource: integrates a file system directory (read only)
    \end{itemize}
\end{frame}

%%
\begin{frame}[fragile]
    \frametitle{SeisHub: Advantages}
    Technical:
    \begin{itemize}
        \item Sharing data over the network, but no firewall problems (HTTP / HTTPS)
        \item License free, open source, internet standards
        \item Platform independent
        \item Most basic client is a standard browser
        \item XML:
        \begin{itemize}
            \item Data validation on upload (XML schemas)
            \item Data transformation on request (XML stylesheets)
        \end{itemize}
        \item Querying: SQL or XPath
    \end{itemize}
    Scientist:
    \begin{itemize}
        \item May modify there data provided as XML document at any time without corrupting the underlying database
        \item May dynamically add or delete search indexes, schemas and stylesheets
    \end{itemize}
\end{frame}

%%
\begin{frame}[fragile]
    \frametitle{SeisHub: Disadvantages}
    Technical:
    \begin{itemize}
        \item Slower than ''common'' solutions
        \begin{itemize}
            \item XML parsing during validation and indexing
            \item Data overhead (XML verbosity)
        \end{itemize}
        \item Infrastructure
    \end{itemize}
    Scientist:
    \begin{itemize}
        \item Seismologist != IT nerds
    \end{itemize}
\end{frame}

\end{document}
