\documentclass[10pt]{article}
\usepackage[utf8x]{inputenc}
\usepackage{textcomp}
\usepackage[left=1cm, bottom=1cm, top=1cm,right=1cm,nohead,nofoot]{geometry}

% Python listing setup

\usepackage{color}
\usepackage[procnames]{listings}
\usepackage{textcomp}
\usepackage{setspace}
\usepackage[]{xcolor}
\definecolor{gray}{gray}{0.5}
\definecolor{green}{rgb}{0,0.5,0}
\definecolor{lightgreen}{rgb}{0,0.7,0}
\definecolor{purple}{rgb}{0.5,0,0.5}
\definecolor{darkred}{rgb}{0.7,0,0}

\lstnewenvironment{python}[1][]{
\lstset{
% Escape with funnyeyes.
escapeinside={(*@}{@*)},
language=python,
basicstyle=\ttfamily\small,
stringstyle=\color{green},
showstringspaces=false,
alsoletter={1234567890},
otherkeywords={\ , \}, \{},
keywordstyle=\color{blue},
emph={access,and,as,break,class,continue,def,del,elif,else,%
except,exec,finally,for,from,global,if,import,in, is,%
lambda,not,or,pass,print,raise,return,try,while,assert},
emphstyle=\color{orange}\bfseries,
emph={[2]self},
emphstyle=[2]\color{gray},
emph={[4]ArithmeticError,AssertionError,AttributeError,BaseException,%
DeprecationWarning,EOFError,Ellipsis,EnvironmentError,Exception,%
False,FloatingPointError,FutureWarning,GeneratorExit,IOError,%
ImportError,ImportWarning,IndentationError,IndexError,KeyError,%
KeyboardInterrupt,LookupError,MemoryError,NameError,None,%
NotImplemented,NotImplementedError,OSError,OverflowError,%
PendingDeprecationWarning,ReferenceError,RuntimeError,RuntimeWarning,%
StandardError,StopIteration,SyntaxError,SyntaxWarning,SystemError,%
SystemExit,TabError,True,TypeError,UnboundLocalError,UnicodeDecodeError,%
UnicodeEncodeError,UnicodeError,UnicodeTranslateError,UnicodeWarning,%
UserWarning,ValueError,Warning,ZeroDivisionError,abs,all,any,apply,%
basestring,bool,buffer,callable,chr,classmethod,cmp,coerce,compile,%
complex,copyright,credits,delattr,dict,dir,divmod,enumerate,eval,%
execfile,exit,file,filter,float,frozenset,getattr,globals,hasattr,%
hash,help,hex,id,input,int,intern,isinstance,issubclass,iter,len,%
license,list,locals,long,map,max,min,object,oct,open,ord,pow,property,%
quit,range,raw_input,reduce,reload,repr,reversed,round,set,setattr,%
slice,sorted,staticmethod,str,sum,super,tuple,type,unichr,unicode,%
vars,xrange,zip},
emphstyle=[4]\color{purple}\bfseries,
upquote=true,
morecomment=[s][\color{lightgreen}]{"""}{"""},
commentstyle=\color{red}\slshape,
literate={>>>}{\bfseries{\textcolor{darkred}{>{>}>}}}3%
         {...}{{\textcolor{gray}{...}}}3,
procnamekeys={def,class},
procnamestyle=\color{blue}\textbf,
framexleftmargin=1mm, framextopmargin=1mm,
rulesepcolor=\color{blue},#1
}}{}



\usepackage{lmodern}

% Meta information.
\title{Questions and Solutions}
\date{}

\begin{document}
  \maketitle


\section*{UTCDateTime Exercises}

\begin{itemize}
\item Calculate the number of hours passed since your birth. \\
    \begin{itemize}
        \item The current date and time can be obtained with \textbf{``UTCDateTime()''}
        \item Optional: Include the correct time zone
    \end{itemize}


\begin{python}
from obspy.core import UTCDateTime
# Set the birth date assuming the timezone at date and place of birth was UTC+1.
birth = UTCDateTime("1980-01-02T03:04:05+01:00")
current_time = UTCDateTime()
print "Hours passed since birth:", (current_time - birth) / 3600.0
\end{python}



\item Get a list of 10 UTCDateTime objects, starting yesterday at 10:00 with a spacing of 90 minutes.

\begin{python}
from obspy.core import UTCDateTime

yesterday = UTCDateTime() - 86400
yesterday_ten_oclock = UTCDateTime(yesterday.year, yesterday.month, yesterday.day, 10)
time_list = [yesterday_ten_oclock + _i * 90 * 60 for _i in range(10)]
\end{python}



\item The first session starts at 09:00 and lasts for 3 hours and 15
    minutes. Assuming we want to have the coffee break 1234 seconds and
    5 microseconds before it ends. At what time is the coffee break?

\begin{python}
from obspy.core import UTCDateTime

session_start = UTCDateTime(2012, 9, 7, 9)
session_end = session_start + 3 * 3600 + 15 * 60
coffee_break = session_end - 1234.000005
print "Coffee at %s" % coffee_break
\end{python}




\item Assume you had your last cup of coffee yesterday at breakfast.
    How many minutes do you have to survive with that cup of coffee?

\begin{python}
from obspy.core import UTCDateTime

yesterday = UTCDateTime() - 86400
yesterday_breakfast = UTCDateTime(yesterday.year, yesterday.month, yesterday.day, 8)
now = UTCDateTime()
hours_without_coffee = (now - yesterday_breakfast) / 3600
print "Hours without coffee:", hours_without_coffee
\end{python}

\end{itemize}

\newpage
\section*{Waveform Exercises}
\subsubsection*{Trace Exercise 1}
    \begin{itemize}
        \item Make a trace with all zeros (e.g. \textit{numpy.zeros(200)}) and an ideal pulse at the center
        \item Fill in some station information (\textit{network, station})
        \item Print trace summary and plot the trace
        \item Change the sampling rate to 20 Hz
        \item Change the \textit{starttime} to the start time of this session
        \item Print the trace summary and plot the trace again
    \end{itemize}

    \begin{python}
from obspy.core import Trace, UTCDateTime
import numpy as np

tr = Trace(data=np.zeros(200))
tr.data[100] = 1
tr.stats.network = "AB"
tr.stats.station = "CDE"
print tr
tr.plot()

tr.stats.sampling_rate = 20
tr.stats.starttime = UTCDateTime()
print tr
tr.plot()
    \end{python}

\subsubsection*{Trace Exercise 2}

\begin{itemize}
        \item Use \textit{tr.filter(...)} and apply a lowpass filter with a corner frequency of 1 second.
        \item Display the preview plot, there are a few seconds of zeros that we can cut off.
        \item Use \textit{tr.trim(...)} to remove some of the zeros at start and at the end.
\end{itemize}

\begin{python}
tr.filter("lowpass", freq=1.0)
tr.plot()
tr.trim(tr.stats.starttime + 3, tr.stats.endtime - 3)
tr.plot()
\end{python}


\subsubsection*{Trace Exercise 3}
\begin{itemize}
        \item Scale up the amplitudes of the trace by a factor of 500
        \item Make a copy of the original trace
        \item Add standard normal gaussian noise to the copied trace (use \textit{numpy.random.randn(..)})
        \item Change the station name of the copied trace
        \item Display the preview plot of the new trace
\end{itemize}


\begin{python}
tr.data *= 500.0
tr.plot()
tr2 = tr.copy()
tr2.data += np.random.randn(len(tr2)) * tr.data.mean()
tr2.stats.station = "ABC"
tr2.plot()
\end{python}

\newpage

\subsubsection*{Stream Exercise}
    \begin{itemize}
        \item Read the example earthquake data into a stream object (\textit{read()} without arguments)
        \item Print the stream summary and display the preview plot
        \item Assign the first trace to a new variable and then remove that trace from the original stream
        \item Print the summary for the single trace and for the stream
    \end{itemize}

\begin{python}
from obspy.core import read
st = read()
print st
st.plot()
tr = st[0]
st.remove(tr)
print tr
print st
\end{python}


\section*{obspy.xseed Exercise}
    \begin{itemize}
        \item Read the \textbf{BW.FURT..EHZ.D.2010.005} waveform example file.
        \item Cut out some minutes of interest.
        \item Read the \textbf{dataless.seed.BW\_FURT} SEED file.
        \item Correct the trimmed waveform file with the poles and zeros from
            the dataless SEED file using \textit{st.simulate()}. This will,
            according to the SEED convention, correct to $m/s$.
        \item (Optional) Read the file again and convert to $m$ by adding an
            extra zero. Choose a sensible waterlevel.
        \item (Optional) Convert the SEED file to XSEED, edit some values and
            convert it back to SEED again. This requires some knowledge of the
            general SEED file structure.
    \end{itemize}


\begin{python}
from obspy.core import read
from obspy.xseed import Parser

st = read("./BW.FURT..EHZ.D.2010.005")
st.trim(st[0].stats.starttime, st[0].stats.starttime + 90)
p = Parser("./dataless.seed.BW_FURT")
paz = p.getPAZ(st[0].id)

st.simulate(paz_remove=paz)

st = read("./BW.FURT..EHZ.D.2010.005")
st.trim(st[0].stats.starttime, st[0].stats.starttime + 90)
paz["zeros"].append(0 + 0j)
st.simulate(paz_remove=paz)
st.plot()
\end{python}


\newpage

\section*{Event Exercise}

    \begin{itemize}
        \item Read the \textbf{example\_catalog.xml} file.
        \item Plot the events.
        \item Print the resulting Catalog object and filter it, so it only contains events with a magnitude larger then 7.
        \item Now assume you did a new magnitude estimation and want to add it
            to one event. Create a new magnitude object, fill it with some
            values and append it to magnitude list of the largest event.
        \item Write the Catalog as a QuakeML object.
    \end{itemize}


\begin{python}
from obspy.core.event import readEvents
cat = readEvents("example_catalog.xml")

cat.plot()
print cat
print cat.filter("magnitude > 7")

from obspy.core.event import Magnitude
mag = Magnitude()
mag.mag = 6.0
cat[0].magnitudes.append(mag)
cat.write("new_events.xml", format="quakeml")
\end{python}

\end{document}
